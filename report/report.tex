\documentclass[a4paper, 12pt]{article}
\usepackage{graphicx}
\usepackage{listings}

\setlength\parindent{24pt}

\lstset{language=python,breaklines=true, frame=single}

\begin{document}
\begin{figure}
    \centering
    \includegraphics[width=1\textwidth]{Logo}
\end{figure}

\title{Project Report}
\author{Manwel Bugeja}
\date{\today}
\maketitle
  
\tableofcontents
\newpage

%\section{Introduction}
%This is an intro. \cite{lowhighlevelevents}
%That was a citation.

\section{Question 1}
\subsection{High level method overview}
\subsubsection{Connection}
A Unix domain socket server and client where created in python. TLS wrappers from the SSL library were used to enable communications over a secure channel. Socket creation and wrapping is shown in listing \ref{lst:tls-sockets}

\begin{lstlisting}[caption={Socket creation and TLS wrapping}\label{lst:tls-sockets}, basicstyle=\ttfamily, frame=single, language=python]
client = create_connection((ip, port))
tls = context.wrap_socket(client, server_hostname=hostname)
\end{lstlisting}

The server has two functions to handle the two types of applications, these being \textit{handle\_application()} and \textit{handle\_client()}.





\section{Question 2}
\section{Question 3}
\section{Question 4}

\bibliographystyle{abbrv}
 \bibliography{references}

\end{document}
